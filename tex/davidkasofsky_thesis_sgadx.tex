\documentclass{article}

\usepackage[utf8]{inputenc}

\usepackage[T1]{fontenc}

\usepackage{geometry}
\geometry{a4paper}

\usepackage[english]{babel}

\usepackage{setspace}
\doublespacing

\usepackage{amsfonts}

\usepackage{amsmath}

\usepackage{bm}

\usepackage{tikz}
\usetikzlibrary{trees}

\title{Signaling Game Ad Exchange}

\author{David Kasofsky}

\date{May 1, 2016}

\begin{document}

\maketitle

\newpage

\tableofcontents

\newpage

\section{Introduction}

TODO

\newpage

\section{Signaling Games}

TODO: discuss crawford \& sobel

\subsection{Basic Model}

A Signaling Game is a two player game of information asymmetry. The sender $S$ has a secret type $t \in T$ unknown to the receiver $R$. $S$ sends a signal $\sigma \in \Sigma$ to $R$ who then performs an action $\alpha \in A$. Both players receive payoffs depending on the action $\alpha$, signal $\delta$, and secret type $t$.

\begin{equation}
	S = (t_S, \pi_s, \Sigma, \mu_S)
\end{equation}

\begin{equation}
	R = (t_R, \pi_R, A, \mu_R)
\end{equation}

\begin{equation}
	S \overset{\sigma}{\longrightarrow} R : \alpha
\end{equation}

\begin{equation}
	\mu_S = \langle t_S, \delta \rangle + \alpha
\end{equation}

\begin{equation}
	\mu_R = \alpha \langle t_S, t_R \rangle
\end{equation}

\begin{equation}
	\Sigma = \lbrace \lambda t_S + (1-\lambda)\pi_S(R) \rbrace
\end{equation}

\begin{equation}
	A = \lbrace 1_{\langle t_R, \delta \rangle > \theta} \rbrace
\end{equation}

Such games are found to resemble interactions in nature, perhaps most famously described by Grafen \cite{grafen1} in his formalization of ideas from Zahavi \cite{zahavi1}. Spence \cite{spence1} gives an example of how such games can model the hiring process.

Consider the following game:simplified Poker. The sender has a winning hand or a losing hand and can send one of two messages: Bet or Check. The receiver can either Call or Fold.

To Bet signals a winning hand while to Check signals a losing one. The game is interesting because the sender may bluff by betting when he has a losing hand. 

With a few examples to build our intuition, we will shortly give a mathematical description of basic signaling games. We say basic here because we will extend this model in later games. First of all, a signaling game is a mathematical game like those studied in game theory. Game theory is generally interested in finding optimal strategies in multiplayer games. A game theoretical analysis almost always starts with assumptions about the goals, abilities, and rationality of the players, perhaps most importantly that the players will all play in such a way to maximize their score in the game. One can imagine player $i$'s score to be the output of some function $u_i$ and so $i$ will play so as to maximize $u_i$. This is a nice way to address arbitrary player preferences in complex or realistic games as such preferences are ultimately expressed by the value of $u_i$. 

These assumptions limit players' playable strategies, e.g. a player will not choose strategy A if it gives a lower score than strategy B. This brings us to the cornerstone of game theory: Nash equilibrium. A Nash equilibrium is a strategy assignment for each player such that each player's score is maximized given the strategies of the others. Thus a Nash Equilibrium is akin to a solution of a game since it gives each player a strategy which cannot be improved given the strategies of the other players. minimum of joint loss?

It is important to note the difference between pure and mixed strategies. A pure strategy is deterministic while a mixed strategy is probabilistic. Recall the Simple Poker game from above. A pure strategy for the sender is to bet winning hands and check losing hands. The best response for the receiver is then to fold to all bets and call all checks. This pair of strategies is not a Nash equilibrium because the sender's best response to the receiver's strategy is to bet losing hands as well, since the receiver will fold to all bets. 

An example of a mixed strategy is to always bet a winning hand and bet a losing hand half of the time. This time, however, we have a hiccup in choosing the receiver's best response strategy. Suppose the receiver knows the sender's strategy, i.e. that the sender is betting half of his losing hands. The receiver still does not know how often the sender is dealt a losing hand and so we need another tool in order to find the receiver's best response.

This brings us to the concept of Bayesian games. Signaling games are Bayesian games and thus so is the Simple Poker game. A Bayesian game is a game in which some information is hidden from the players. Imagine a set $\Omega$, elements of which represent possible gamestates. In the Simple Poker game, for example, $\Omega = \{\text{Winning Hand}, \text{Losing Hand}\}$. When a Bayesian game is initialized, an element of $\Omega$ is chosen and each player $i$ is assigned some probability distribution $\pi_{i}(\cdot)$ over $\Omega$, i.e. a prior belief over the possible gamestates. In the case of the signaling game, the sender $S$ is assigned $\pi_S(\cdot)$ which gives all mass to the correct gamestate.

Now we can address the receiver's best response in the Simple Poker game when the sender plays a mixed strategy. Recall that even if we knew how much the sender was bluffing, we could not choose a best response without also considering how often the sender had a winning hand. Let us now suppose that the receiver believes the sender has a winning hand half of the time, i.e. $\pi_R(\text{Losing Hand}) = 0.5$. Now it is straightforward to pick the best response strategy. Instead of folding to all bets as in the pure strategy response, the receiver should now call one third of the bets the sender makes. This is because if the sender is betting all his winning hands and half his losing ones, he will bet three quarters of all his hands and one third of those bets will be with losing hands. Naturally the values from the above payoff matrix were relevant in this reasoning but were chosen for our  at this point.

With these concepts in hand, we can concisely describe the basic signaling game. There are two players, $S$ and $R$. $S$ has some type $t \in T$ which is known to $S$ but unknown to $R$. $T$ is analogous to the set $\Omega$ from before. Each player also has a prior $ \pi_i(dcot)$ over $T$. One can imagine $\pi_{S}$ assigns probability 1 to $t$ and $\pi_{R}$ does not, e.g. $\pi_{R}$ could assign uniform probability to each $t \in T$. $S$ chooses a message $m$ from a set $M$ which is sent to $R$. Considering $m$ in light of $\pi_R$, $R$ chooses an action $a$ from a set $A$. Each player then receives a score according to functions $u_R$ and $u_S$ : $T \times M \times A \to \mathbb{R}$.
A strategy for S is then a function $\mu_{S}: T \times M \to [0,1]$ such that $\forall t: \sum_{m \in M} \mu_{S}(t, m) = 1$, i.e. assigns a probability to each message for all types. A strategy for R is a similarly $\mu_{R} : M \times A \to [0,1]$ such that $\forall m: \sum_{a \in A} \mu_{R}(m,a) = 1$.

A strategy for S is then a function $\mu_{S}: T \times M \to [0,1]$ such that $\forall t: \sum_{m \in M} \mu_{S}(t, m) = 1$, i.e. assigns a probability to each message for all types. A strategy for R is a similarly $\mu_{R} : M \times A \to [0,1]$ such that $\forall m: \sum_{a \in A} \mu_{R}(m,a) = 1$.

\subsection{Equilibria}

Three types of Nash equilibria exist in signaling games:

\begin{enumerate}
\item Separating Equilibria, in which each type of sender send different signals. That is:

$\forall t, t' \in T, \exists M_{t} \subset M$ with $\mu_S(t,m) > 0 \Rightarrow m \in M_{t}$ and $t \ne t' \Rightarrow M_{t} \cap M_{t'} = \emptyset$.

\item Pooling Equilibria, in which all types send the same message. That is: $\exists  m \in M, \forall t \in T : \mu_S(t,m) = 1$.

\item Mixed Equilibria, which fall into neither of the previous two categories. Senders of differing types may send the same messages but may not as well. 
\end{enumerate}

Because signaling games are games of information asymmetry, we can interpret strategies and equilibria in terms of deception. The informed player, S, can deceive the uninformed player R. Consider the Simple Poker game once more: to sometimes bet with losing hands is a deceptive strategy that exploits the receiver's ignorance. 

 For instance, a separating equilibrium suits games in which the sender and receiver have a common interest and so the sender does not necessarily wish to be deceptive. Rather the sender wants to confer the maximum amount of information about the hidden type via the signal.

In a separating equilibrium, the receiver is able to deduce the sender's type from signal. Since the sender's utility function $u_{S}$ and possible types $T_S$ are common knowledge, the receiver can deduce which signal would be sent by each type. This makes the receiver's choice of action straightforward and there is no deception by the sender.

In a pooling equilibrium, the sender confers no information about the hidden type as all senders send the same signal. In this case the receiver's strategy is independent of the signal and depends only on $\pi_{ R}$. This is maximally deceptive but so obviously as to render the signal meaningless.

\subsection{Costly Signaling}

Costly signaling is an important feature of many signaling games. Suppose $G$ is a signaling game with $M$ a real interval and consider $u_{S}(t,s,a)$.

\subsection{Iterated Signaling Games}

\subsection{Recommenders and Verifiers}

\newpage

\section{Simplified Game}

an iterative Bayesian game modeling online ad interaction

Here we model online advertising via signaling games.  Consumers want to interact with ads that suit them.

An advertiser $S$ sends an ad $\delta$ to a consumer $R$. The consumer clicks or not, represented by the binary variable $\alpha$. Both $S$ and $R$ have types $t_S$ and $t_R$ respectively. Ads are types are represented by unit vectors in $\mathbb{R}^N$.

Both the 



An advertiser $S$ sends a message to a consumer $R$. Both $S$ and $R$ have types $t_S$ and $t_R$ respectively. 

\section{Ad Exchange Simulation}

\subsection{Agent-based Model}

\subsection{Simulation Details}

\subsection{Implementation}

\section{Simulation Results}

\subsection{Summary}

\section{References}

\end{document}

