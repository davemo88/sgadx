
\documentclass{article}

\usepackage[utf8]{inputenc}

\usepackage[T1]{fontenc}

\usepackage{geometry}
\geometry{a4paper}
 
\usepackage{float}
\restylefloat{table}

\usepackage[english]{babel}

\usepackage{setspace}
\doublespacing

\usepackage{amsfonts}
\usepackage{amsmath}

\usepackage{bm}

\usepackage{tikz}
\usetikzlibrary{trees}

\usepackage{algorithm}
\usepackage{algorithmic}

\usepackage{todonotes}

\title{Master's Thesis: Ads via Signaling Games}

\author{David Kasofsky}

\date{May 1, 2016}

\begin{document}

\maketitle

\newpage

\tableofcontents

\newpage

\section{Introduction}

The goal of this paper is to demonstrate an application of game theory and machine learning in advertising. In particular, we model advertising interaction via signaling games and imagine the players as machine learning algorithms. Advertising is an appealing domain as it is a driving force in the success of the internet and motivates many challenging computer science problems.

\section{Advertising}
Advertising has powered the growth of the internet. As the internet becomes the primary ad medium\cite{iab1}, advertising strategy, measurement, and implementation increasingly rely on big data, machine learning, and cloud computing. However twenty first century advertising has also motivated concerns about privacy.

Much of the activity on the internet generates data potentially useful in advertising strategy, e.g. website analytics data or user demographic data. There is an incentive to store an enormous amount of data for use by advertisers. In turn, advertisers need data storage and processing frameworks which are suited to this setting, e.g. MapReduce\cite{mapreduce1}. All of this new data has generated interest in techniques that can make sense of it, e.g. targeted advertising \cite{displayadsml1}, ad evaluation \cite{abhishek2012media}, and real-time bidding algorithms \cite{yuan2014survey}. Tech giants Google and Facebook are built on the back of advertising \cite{googlerevenue}.\todo{could also mention ranking and search engines}



\subsection{Consumer Data}
In particular, advertisers want consumer data. 

Advertisers' hunger for consumer data has not only inspired techniques for data storage and processing but also data generation and collection. There are two privacy concerns here. The first is that businesses may be observing and storing personal data that the consumer does not wish to share or persist. The second is that businesses may treat the consumer differently based on personal data to the possible detriment of the consumer. In the worst case, the business is able to unduly exploit the consumer.

For instance, a consumer may be offered a product at a higher price depending on their personal data. In an extreme example they might be denied a service altogether. Imagine Joe has a minor condition he doesn't know about. He applies for health insurance and is told he must pay a higher rate because of the condition. When he asks how he is known to have the condition, he learns his DNA matched DNA collected anonymously by a subway turnstyle which was analyze d and found to have the condition. While this is clearly a bit of hyperbole, virtually every website aspires to be like that turnstyle and scrape the most data they can off the user.x

\subsection{Deception}
this section sucks

Access to consumers' personal data gives advertisers an information advantage, especially since consumers rarely if ever get a peek at the internal data of an advertiser. For example: consumers don't know how valuable they are to advertisers but advertisers can estimate the value of each consumer. This advantage can be used deceptively by advertisers, e.g. to play to the consumers' tastes. This can be as simple as sending different ads to different types of users.

This deception can be harmful to consumers. If advertisers are maximally deceptive and forego honest representation of their products to instead appeal to consumers, consumers will eventually learn not to trust advertisers. Proverb: You can shear a sheep every so often but only skin him once. The better advertisers can predict consumer types, the better they can select appealing ads. The capability to engage consumers who would normally not consider the advertiser's products naturally prompts the advertiser to wonder how far one might.....

However, as we know \cite{phishing for phools}, consumers are not equipped to consistently make the best rational decisions and so this unduly exploitation may continue.

\section{Signaling Games}

A Signaling Game is a two-player game of information asymmetry. The sender $S$ has a secret type $\tau \in T$ unknown to the receiver $R$. $S$ sends a signal $\sigma$ to $R$ who then performs an action $\alpha$. Players receive payoffs depending on the action $\alpha$, signal $\sigma$, and secret type $\tau$. This statement of the game is inspired by Sobel\cite{sobel1}.

Signaling games appear in many interesting places: in mate selection by Zahavi\cite{zahavi1} and Grafen\cite{grafen1}, in job markets by Spence[3], in TODO by Crawford\cite{crawford1} and in cybersecurity by Casey\cite{casey1}\cite{casey2}\cite{casey3}.


\subsection{Simple Poker}

Consider a simplified version of two-player poker. The sender $S$ is dealt either a winning hand or a losing hand and may check or bet. The receiver $R$ then may fold or call. The payoff matrix in \ref{simplepokerpayoffs} shows the poker-inspired messages, actions, and payoffs.

\begin{table}[H]
	\centering
	\caption{Simple Poker Payoffs}
	\label{simplepokerpayoffs}
	\begin{tabular}{ll|l|l|}
		\cline{3-4}

Suppose $S$ has a winning hand. Then bet dominates check for $S$ and fold dominates call for $R$. However if $S$ has a losing hand then the opposite is true. Say $S$ bets all winners 
		&       & \textbf{Fold} & \textbf{Call} \\ \hline
		\multicolumn{1}{|l|}{$\tau=\textbf{Winner}$} & \textbf{Check} & (1,-1)  & (1,-1)  \\ \cline{2-4}
		\multicolumn{1}{|l|}{}        & \textbf{Bet}   & (1,-1)  & (2,-2)  \\ \hline
		\multicolumn{1}{|l|}{$\tau=\textbf{Loser}$}  & \textbf{Check} & (1,-1)  & (-1,1)  \\ \cline{2-4}
		\multicolumn{1}{|l|}{}        & \textbf{Bet}   & (1,-1)  & (-2,2) \\ \hline
	\end{tabular}
\end{table}and checks all losers. Call this strategy $h_1$. If $S$ plays $h_1$, then $R$'s best response is to fold to all bets and call all checks ($g_1$). However $(h_1, g_1)$ is clearly not a Nash equilibrium of the game since $S$ has a different best response to $g_1$: bet every hand ($h_2$).

In considering $R$'s best response to $h_2$, we stumble across several interesting points. The first is that if $S$ bets every hand, i.e. the sender always sends the same signal regardless of the secret type, then $R$ cannot infer any information about the secret type from the signal. If this is the case, then $R$ has to make a guess about the sender's type $\tau$. For instance $R$ may guess that $S$ has a winning hand half the time, i.e. $P(\tau = \textbf{Winner}) = \frac{1}{2}$. Then $R$ could randomize uniformly between calling and folding ($g_2$). When in fact $p = \frac{1}{2}$ then $(h_2, g_2)$ is a Nash equilibrium.

Second, poker is usually played as a repeated game. The Nash equilibria will depend on the true probability $p$ that $S$ has a winning hand. For example,

For a more formal statement of the game, we define $\left(T, \Sigma, A, \mu_S, \mu_R, \right)$, where:
\begin{itemize}
	\item $T = \lbrace 0, 1 \rbrace$ is the type set. To begin we chose some $\tau \in T$ as the type of $S$. 0 corresponds to loser and 1 to winner.
	\item $\Sigma = \lbrace 0, 1 \rbrace$ is the signal set. 0 corresponds to check and 1 to bet.
		\item $A = \lbrace 0, 1 \rbrace$ is the action set. 0 corresponds to fold and 1 to call.
	\item $\mu_S: T \times \Sigma \times A \longrightarrow \mathbb{R}$ is the sender's utility function:
		\begin{equation}
\mu_S(\tau, \sigma, \alpha) = \tau(1+\sigma\alpha) + (1-\tau)(1-\sigma\alpha-\alpha)
		\end{equation}
	\item $\mu_R: T \times \Sigma \times A \longrightarrow \mathbb{R}$ is the receiver's utility function:
		\begin{equation}
\mu_R(\tau, \sigma, \alpha) = \tau(-\sigma\alpha) + (1-\tau)(\sigma\alpha+\alpha)
		\end{equation}
\end{itemize}

\noindent These are the utility functions used to generate Table \ref{simplepokerpayoffs} above.

\section{Learning and Games}
    

\subsection{Bayesian Games}

\subsection{Regret}

\subsection{Weighted Majority}

\subsection{Swap Regret}

\subsection{Correlated Equilibrium}

\section{Advertising Interaction via Signaling Games}

NOTE: section 1, working title "Advertising", will introduce all the lingo I use here

In this twist on the signaling game, we depict the bread and butter of online advertising: interaction with ad content chosen using consumer data. The sender $S$ is an advertiser and the receiver $R$ a consumer.

Let us begin this time with a more formal statement. First we have the sender $S = (\tau_S, \pi_S, \Sigma, \mu_S, H)$, where:
\begin{itemize}
	\item $\tau_S \in T_S = U^D$. $\tau_S$ is the type of $S$. $T_S$ is the type space of $S$.
	\item $\pi_S$ is a probability distribution over $T_R$, the type space of $R$.
	\item Signal space $\Sigma = U^D$.
	\item Utility function $\mu_S: T_S \times \Sigma \times A \rightarrow \mathbb{R}$ is defined as:
	\begin{equation}
		\mu_S(\tau_S, \sigma, \alpha) = \langle \tau_S, \sigma \rangle + \alpha
	\end{equation}
	\begin{equation}
		\mu_S = \langle \tau_S, \sigma \rangle + \alpha \langle \sigma, \tau_R \rangle
	\end{equation}
	\item Strategy set $H$ is defined as:
	\begin{equation}
		H = \lbrace h | h(\tau_S, \pi_S; \lambda) = \frac{\lambda \tau_S + (1-\lambda)\underset{x\sim\pi_S}{E}\lbrack x \rbrack}{Z}; 0 \le \lambda \le 1 \rbrace
	\end{equation}
\end{itemize}

\noindent Next we have the receiver $R = (\tau_R, \pi_R, A, \mu_R, G)$, where:
\begin{itemize}
	\item $\tau_R \in T_R = U^D$.
	\item $\pi_R(\cdot|\sigma)$ is a conditional distribution over $T_S$.
	\item $A = \lbrace 0,1 \rbrace$. 0 corresponds to no click, 1 to click.
	\item Utility function $\mu_R: T_S \times T_R \times A \rightarrow \mathbb{R}$ is defined as:
	\begin{equation}
		\mu_R(\tau_S, \tau_R, \alpha) = \alpha \langle \tau_S, \tau_R \rangle
	\end{equation}
	\begin{equation}
		\mu_R = \langle \tau_R, \sigma \rangle + \alpha \langle \sigma, \tau_S \rangle
	\end{equation}
	\item Strategy set $G$ is defined as:
	\begin{equation}
		G = \lbrace g | g(\tau_R, \sigma; \theta) = 1_{\langle \tau_R, \sigma \rangle > \theta}; 0 \le \theta \le 1 \rbrace
	\end{equation}
\end{itemize}

\noindent The first twist in this game is that both $S$ and $R$ have types $\tau_S$ and $\tau_R$ respectively. These types are $D$-dimensional unit vectors. Each component of these vectors can be interpreted as an affinity for a product category. We can compare the types of $S$ and $R$ to determine the suitability of the advertiser's products for the consumer. 

Similarly both $S$ and $R$ have priors $\pi_S$ and $\pi_R$ over the other player's type space. We will express this as a regularization term in the strategy constraining how we will choose the direction and step size for gradient descent

Furthermore, the ad $\sigma$ is also a $D$-dimensional unit vector. Once again we can interpret the components as product category affinities and imagine that the ad promotes certain types of products. The action $\alpha$ is binary, corresponding to whether or not the consumer clicks on the ad.

The sender's utility function $\mu_S$ punishes advertisers for deception, i.e. the advertiser incurs a cost as the ad differs from the advertiser's type. The advertiser is rewarded for clicks, regardless of the type of the consumer.

The receiver's utility function $\mu_R$ is zero unless the receiver clicks. If so $R$ receives utility in proportion to the similarity of $\tau_S$ and $\tau_R$. This is interesting because if the sender clicks on an ad for an unsuitable advertiser, i.e. where $\langle \tau_S, \tau_R \rangle < 0$, they will incur a cost. Since $\tau_S$ is unknown to $R$, $R$ must decide to click based on the signal $\sigma$ and the prior $\pi_R$, just like the standard case.

These utility functions are common knowledge. Thus the consumer knows that $\sigma$ may only stray from $\tau_S$ so far before the advertiser incurs a penalty.

$S$ is limited by the strategy set $H$ to signals which are a normalized convex combination of $\tau_S$ and $\underset{x\sim\pi_S}{E}\lbrack x \rbrack$, the average type according to the prior $\pi_S$. The parameter $\lambda$ more or less corresponds to $S$'s level of deception.

QUESTION: what is corresponding rotation instead of the normalized convex combination?

$R$ is limited by the strategy set $G$ to threshold functions of the inner product $\langle \tau_S, \sigma \rangle$. The parameter $\theta$ determines how conservatively $R$ clicks.

In the iterated game, each player wants to tune their parameter ($\lambda$ for senders and $\theta$ for receivers) to be optimal against the population of opposing players. The simulation will be a sort of parallel stochastic gradient descent in which players are randomly paired to play against each other, execute their current strategy, then receive their payoff. Payoff functions are public knowledge so players are able to make inferences about the distribution of types in the population and thereby pick a direction and learning rate for gradient descent.

Loss in the iterated game is average regret after $\eta$ rounds. We expose the types of the players when we give payoffs but not their identity, i.e. players cannot recognize each other after meeting. By exposing the types we make this into a supervised online learning problem for each player.

\subsection{possible generalizations and future directions}

\begin{itemize}
	\item richer engagement with the ad e.g. higher dimensional
	\item ads needed to be suited to both context (current webpage or content request) and consumer type
\end{itemize}

\section{References}

\bibliographystyle{amsplain}
\bibliography{dk_thesis}

\end{document}
